\documentclass[a4paper, 12pt]{report}
%================================================================
\usepackage{color,amsmath,amsfonts,amssymb,epsfig,hyperref}
\usepackage{graphicx}
\usepackage{dsfont}
\usepackage[latin1]{inputenc}
\usepackage[T1]{fontenc}
\usepackage[english]{babel}
\usepackage{pdfpages}
%====================================
 \textheight 24cm
% \doublespace
  \oddsidemargin -0.5cm
  \evensidemargin +1.5cm
  \textwidth 17cm
 \topmargin -2cm
%============================================
% Figures
%============================================
\newcommand{\figsstit}[2]{
\begin{figure}[hbtp]
\centerline{
    \hbox{ \includegraphics[scale=#2]{#1} }
}
\end{figure}}
%============================================
\newcommand{\figscale}[4]{
\begin{figure}[hbtp]
\centerline{
    \hbox{ \includegraphics[scale=#4]{#1} }
}
\begin{center}
\parbox{14 cm}
{
    \caption{\protect\small\it  {#2}}
    \label {#3}
}
\end{center}

\end{figure}}

%==================================================
\newcommand\algo[1]%
{
    \begin{center} %
    \begin{tabular} {||p{10 cm}l ||}%
    \hline
               #1 &  \\
    \hline
    \end{tabular}
    \vspace{12pt}
    \end{center}
}


%==============================================
\newcommand{\prob}[1]{\mathds{P}\left( #1 \right)}
\newcommand{\esp}[1]{\mathds{E}\left[ #1 \right]}
\newcommand{\var}[1]{\mathrm{var}\left( #1 \right)}
\newcommand{\cov}[1]{\mathrm{cov}\left( #1 \right)}
\newcommand{\diag}[1]{\mathrm{diag}\left( #1 \right)}
\newcommand{\trace}[1]{\mathrm{trace}\left( #1 \right)}
\newcommand{\card}[1]{\left| #1 \right|}
\newcommand{\myemph}[1]{\emph{\color{red}#1}}

%%============================================================================
\def\thesection{\arabic{section}}
%\def\thesubsection{\arabic{section}.\arabic{subsection}}
%\def\thesubsubsection{\arabic{section}.\arabic{subsection}.\arabic{subsubsection}}
%\def\thefigure{\arabic{figure}}
%\def\theequation{\arabic{equation}}
%\def\theexercice{\arabic{exercice}}
%\def\theexample{\arabic{example}}
%\def\theproof{\arabic{proof}}

%===============================================
\newtheorem{property}{Properties}
\newtheorem{remark}{Remark}
\newtheorem{theorem}{Theorem}
\newtheorem{definition}{Definition}
\newtheorem{example}{Example}
\newtheorem{lemme}{Lemme - \thelemme}
\newtheorem{proof}{Proof - \theproof}
\newenvironment{TAB}{\begin{table}[[hbt] \center \leavevmode}{\end{table}}
%%============================================================================
%\renewcommand\arraystretch{1.6}

\def\ua{\underline a}
\def\ub{\underline b}
\def\uB{\underline B}
\def\uH{\underline H}
\def\ur{\underline r}
\def\us{\underline s}
\def\ux{\underline x}
\def\uX{\underline X}
\def\uZ{\underline Z}
\def\utheta{\underline \theta}



\def\tn{\mathrm{TN}}
\def\fn{\mathrm{FN}}
\def\tp{\mathrm{TP}}
\def\fp{\mathrm{FP}}
\def\tpn{\mathrm{tPN}}
\def\tnn{\mathrm{tNN}}
\def\tdn{\mathrm{tDN}}

\def\precision{\mathrm{\color{red}Precision}}
\def\recall{\mathrm{\color{red}Recall}}
\def\fscore{{\color{red}F\mathrm{-score}}}
\def\far{\mathrm{FAR}}
\def\mdr{\mathrm{MDR}}
\def\vdr{\mathrm{VDR}}
\def\ci{\mathrm{CI}}
\def\pfa{P_{\mathrm{FA}}}
\def\pd{P_{\mathrm{D}}}
\def\loc{\mathrm{LOC}}

\def\SNR{\mathrm{SNR}}
\def\crb{\mathrm{CRB}}
\def\fim{\mathrm{FIM}}

\def\auc{\mathrm{AUC}}
\def\aec{\mathrm{aec}}



\def\void{{\small void}}
\def\nomeaning{{\small meaningless}}
\def\unknown{{\small unknown}}
\def\MSC{\mathrm{MSC}}
\def\hMSC{\widehat{\MSC}}%{\MSC}} 
\def\ellk{{k}}
\def\SOI{common signal part }
\def\absGamma{\Phi}

%============== colors ========================
\definecolor{enstrouge}{RGB}{212,65,84}
\definecolor{lightorange}{RGB}{235,226,52}
\definecolor{greennoise}{RGB}{243,42,255}
\definecolor{lightred}{RGB}{255,181,183}
\definecolor{light-grey}{rgb}{0.95,0.95,0.95}
\definecolor{peach}{rgb}{0.98,0.49,0.25}
\definecolor{burntorange}{rgb}{0.79,0.37,0}
\definecolor{light-yellow}{rgb}{1,1,0.92}

\definecolor{light-green}{RGB}{231,255,145}
\definecolor{enstorange}{RGB}{255,214,10}
\definecolor{enstrouge}{RGB}{212,65,84}
\definecolor{grey}{RGB}{204,204,204}
\definecolor{blue}{RGB}{0,0,255}
\definecolor{almost-black}{RGB}{100,100,100}
\definecolor{violet}{rgb}{0.4,0,0.4}
\definecolor{cyan}{RGB}{0,255,255}
\definecolor{magenta}{RGB}{243,42,255}

\def\degree{^{\circ}}
\def\simiid{\stackrel{\mathrm{i.i.d.}}{\sim}}
\def\simind{\stackrel{\mathrm{ind.}}{\sim}}

 
%%%============================================================================
%%\def\thesection{\arabic{section}}
%%\def\thefigure{\arabic{figure}}
%%\def\theequation{\arabic{equation}}
%%\def\theexercice{\arabic{exercice}}
%%\def\theequation{\arabic{exercice}.\arabic{equation}}
%%%============================================================================
%%\newcounter{auxiliaire}
%%%%%%%%% comment
%%\setcounter{auxiliaire}{\theenumi}
%%\end{enumerate}
%% TEXTE
%%\begin{enumerate}
%%\setcounter{enumi}{\theauxiliaire}
%%%============================================================================

 \bibliographystyle{plain} 

\begin{document}
 \sloppy
%=======================================================
%=======================================================
\section{Introduction}
A 3D seismic device measures the three components of an acceleration vector in 3 specific directions attached to the device. These 3 directions form a trihedron almost orthogonal but not exactly. We denote $v_{u,1}$,  $v_{u,2}$,  $v_{u,3}$ the unitary vectors of these 3 directions of the seismic device under test (SUT), and  $v_{r,1}$,  $v_{r,2}$,  $v_{r,3}$ those of the reference system (SREF). It follows that the 3 signals on SUT write:
\begin{eqnarray*}
\left\{
\begin{array}{rcl}
x_{u,1}(t)&=&h_{u,1}(t)\star (v_{u,1,1}g_{x}(t)+v_{u,1,2}g_{y}(t)+v_{u,1,3}g_{z}(t))
\\
x_{u,2}(t)&=&h_{u,2}(t)\star (v_{u,2,1}g_{x}(t)+v_{u,2,2}g_{y}(t)+v_{u,2,3}g_{z}(t))
\\
x_{u,3}(t)&=&h_{u,3}(t)\star (v_{u,3,1}g_{x}(t)+v_{u,3,2}g_{y}(t)+v_{u,3,3}g_{z}(t))
\end{array}
\right.
\end{eqnarray*}
and similar equations for the SREF. In frequency domain we get:
\begin{eqnarray*}
\left\{
\begin{array}{rcl}
X_{u,1}(f)&=&H_{u,1}(f) (v_{u,1,1}G_{x}(f)+v_{u,1,2}G_{y}(f)+v_{u,1,3}G_{z}(f))
\\
X_{u,2}(f)&=&H_{u,2}(f) (v_{u,2,1}G_{x}(f)+v_{u,2,2}G_{y}(f)+v_{u,2,3}G_{z}(f))
\\
X_{u,3}(f)&=&H_{u,3}(f) (v_{u,3,1}G_{x}(f)+v_{u,3,2}G_{y}(f)+v_{u,3,3}G_{z}(f))
\end{array}
\right.
\end{eqnarray*}
More concisely we write
\begin{eqnarray*}
X_{u}(f)&=&H_{u}(f)V_{u}G(f)
\end{eqnarray*}
where 
\begin{eqnarray*}
H_{u}(f)&=&
\begin{bmatrix}
H_{u,1}(f)&0&0
\\
0&H_{u,2}(f)&0
\\
0&0&H_{u,3}(f)
\end{bmatrix}
\end{eqnarray*}
and where $V_{u}$ is the square matrix whose entries are $v_{u,k}$. Because the columns of $V_{u}$ are normalized, $V_{u}$ depends on 6 free parameters. A possible parametrization is:
\begin{eqnarray}
\label{eq:parametricformofV}
V_{u}(\phi)&=&
\begin{bmatrix}
\cos(a_{1})\cos(e_{1})&\sin(a_{1})\cos(e_{1})&\sin(e_{1})
\\
\cos(a_{2})\cos(e_{2})&\sin(a_{2})\cos(e_{2})&\sin(e_{2})
\\
\cos(a_{3})\cos(e_{3})&\sin(a_{3})\cos(e_{3})&\sin(e_{3})
\end{bmatrix}
\end{eqnarray}

In our protocol, the  reference system (SREF) is a portable seismic device whose the trihedron  is perfectly known. 
Therefore we have in the absence of noise:
\begin{eqnarray}
\label{eq:signalmodel}
\left\{
\begin{array}{rcl}
X_{u}(f)&=&H_{u}(f)V_{u}G(f)
\\
X_{r}(f)&=&H_{r}(f)V_{r}G(f)
\end{array}
\right.
\end{eqnarray}

The objective is to estimate the matrix $V_{u}$, knowing $H_{r}(f)$ and observing $X_{u}(f)$ and $X_{r}(f)$.
More specifically we can estimate the product $H_{u}(f)V_{u}$, but in the absence of a priori knowledge it is impossible to solve separately $H_{u}(f)$ and $V_{u}$.


%=====================================
%=====================================
\section{Resolution w.r.t. $V_{u}$}
%=====================================
In this section we assume that we know the response $H_{u}(f)$ of the SUT. Then solving the second equation of \eqref{eq:signalmodel} w.r.t. $G(f)$, we get:
\begin{eqnarray*}
X_{u}(f)&=&H_{u}(f)V_{u}V_{r} ^{-1}H_{r}^{-1}(f)X_{r}(f)
\end{eqnarray*}
It is worth to notice that integrating w.r.t. $f$ leads
\begin{eqnarray*}
\sum H_{u}^{-1}(f)X_{u}(f)X_{u}^{H}(f)&=&V_{u}V_{r} ^{-1}\sum H_{r}^{-1}(f)X_{r}(f)X_{u}^{H}(f)
\end{eqnarray*}

We let 
\begin{eqnarray*}
S_{uu}&=&\sum H_{u}^{-1}(f)X_{u}(f)X_{u}^{H}(f)
\end{eqnarray*}
and
\begin{eqnarray*}
S_{ru}&=&\sum H_{r}^{-1}(f)X_{r}(f)X_{u}^{H}(f)
\end{eqnarray*}
and provides the solution:
\begin{eqnarray*}
\hat \phi &=& \arg\min_{\phi} \|S_{uu} - V_{u}(\phi)V_{r}^{-1}S_{ru}\|^{2}
\end{eqnarray*}
which can be solved numerically.
Finally, carrying this estimate in \eqref{eq:parametricformofV} provides an estimate of $V_{u}$.
%=====================================
%=====================================
\section{General resolution}
%=====================================
In this section we assume that the response $H_{u}(f)$ of the SUT depends on a $P$-dimensional parameter $a$.  Then:

\begin{eqnarray*}
[\hat \phi, \hat a] &=& \arg\min_{\phi, a} \|S_{uu}(a) - V_{u}(\phi)V_{r}^{-1}S_{ru}\|^{2}
\end{eqnarray*}
where 
\begin{eqnarray*}
S_{uu}(a) &=&
\sum H_{u}^{-1}(f;a)X_{u}(f)X_{u}^{H}(f)
\end{eqnarray*}
which can be solved numerically with reasonable computational effort (maybe!)

\end{document}



